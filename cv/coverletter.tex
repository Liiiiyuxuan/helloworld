%%%%%%%%%%%%%%%%%%%%%%%%%%%%%%%%%%%%%%%%%
% "ModernCV" CV and Cover Letter
% LaTeX Template
% Version 1.1 (9/12/12)
%
% This template has been downloaded from:
% http://www.LaTeXTemplates.com
%
% Original author:
% Xavier Danaux (xdanaux@gmail.com)
%
% License:
% CC BY-NC-SA 3.0 (http://creativecommons.org/licenses/by-nc-sa/3.0/)
%
% Important note:
% This template requires the moderncv.cls and .sty files to be in the same
% directory as this .tex file. These files provide the resume style and themes
% used for structuring the document.
%
%%%%%%%%%%%%%%%%%%%%%%%%%%%%%%%%%%%%%%%%%

%----------------------------------------------------------------------------------------
%	PACKAGES AND OTHER DOCUMENT CONFIGURATIONS
%----------------------------------------------------------------------------------------

\documentclass[11pt,a4paper,sans]{moderncv} % Font sizes: 10, 11, or 12; paper sizes: a4paper, letterpaper, a5paper, legalpaper, executivepaper or landscape; font families: sans or roman

\moderncvstyle{casual} % CV theme - options include: 'casual' (default), 'classic', 'oldstyle' and 'banking'
\moderncvcolor{grey} % CV color - options include: 'blue' (default), 'orange', 'green', 'red', 'purple', 'grey' and 'black'

\usepackage{lipsum} % Used for inserting dummy 'Lorem ipsum' text into the template

\usepackage[scale=0.75]{geometry} % Reduce document margins
%\setlength{\hintscolumnwidth}{3cm} % Uncomment to change the width of the dates column
%\setlength{\makecvtitlenamewidth}{10cm} % For the 'classic' style, uncomment to adjust the width of the space allocated to your name

%----------------------------------------------------------------------------------------
%	NAME AND CONTACT INFORMATION SECTION
%----------------------------------------------------------------------------------------

\firstname{} % Your first name
\familyname{} % Your last name

% All information in this block is optional, comment out any lines you don't need
\title{Teaching and Research Statement}
\address{Yuxuan (Eason) Li}{}
\mobile{(514) 518 1729}
%\phone{(000) 111 1112}
%\fax{(000) 111 1113}
\email{liiiiyuxuan@gmail.com}
%\homepage{staff.org.edu/~jsmith}{staff.org.edu/$\sim$jsmith} % The first argument is the url for the clickable link, the second argument is the url displayed in the template - this allows special characters to be displayed such as the tilde in this example
%\extrainfo{additional information}
%\photo[70pt][0.4pt]{pictures/picture} % The first bracket is the picture height, the second is the thickness of the frame around the picture (0pt for no frame)
%\quote{"A witty and playful quotation" - John Smith}

%----------------------------------------------------------------------------------------

\begin{document}
\makecvtitle % Print the CV title
%----------------------------------------------------------------------------------------
%	EDUCATION SECTION
%----------------------------------------------------------------------------------------

\section{Previous Research Experience}
I am writing to express my enthusiastic interest in the Research Assistant position for the Digital Products and Platforms Practices at the National University of Singapore - Institute of Systems Science (NUS-ISS), as advertised. With a background in [relevant experience or education], I am excited about the opportunity to contribute to the innovative and dynamic environment at NUS-ISS.

{\hskip 2em} My academic and practical experience has equipped me with a solid foundation in critical thinking, analytical reasoning, and an eye for detail—essential skills for effectively managing and refining course curricula. I am particularly drawn to the hands-on nature of the role, which aligns seamlessly with my passion for product management and commitment to staying abreast of industry trends.

{\hskip 2em} Having explored the comprehensive overview of NUS-ISS's Digital Products and Platforms Practice, I am inspired by the institute's commitment to nurturing digital talent and leading the way in shaping the next curve of digital excellence. The prospect of actively participating in courseware analysis, recommending enhancements, and contributing to the evolution of product management education excites me. I am confident that my ability to manage tasks and timelines, coupled with a proactive approach to learning, will allow me to make meaningful contributions to the team.

{\hskip 2em} Moreover, the emphasis on market and trend analysis resonates with my natural curiosity and dedication to staying informed about the latest advancements in the tech industry. I am eager to leverage my research and documentation skills to evaluate content topics, identify opportunities for improvement, and contribute to the ongoing success of the institute.

{\hskip 2em} I am drawn to NUS-ISS not only for its reputable programs and experienced staff but also for its commitment to providing a hands-on and collaborative learning experience. As an aspiring product professional, I am excited about the prospect of working closely with industry experts, refining blended learning content, and contributing to the institute's continued success in developing digital leaders.

{\hskip 2em} Thank you for considering my application. I am eager to bring my skills and passion for product management to NUS-ISS and contribute to the institute's mission of bridging future opportunities through cutting-edge education.

{\hskip 2em} I look forward to the opportunity to discuss my candidacy further and explore how I can contribute to the success of NUS-ISS. Thank you for considering my application.



{\hskip 2em}
%----------------------------------------------------------------------------------------
\section{Research Goals}

$\bullet{}$ 
INvestigate and research current industry trends in digital product 
management to learn methodologies, tools, and best practices.

$\bullet{}$ 
Explore and analyze existing courseware across modules to identify 
opportunities for streamlining concepts and unifying outcomes for enhanced 
learning experiences.

$\bullet{}$ 
Research major product management educators and communities to assess content topics and trends, contributing to the ongoing evolution of our curriculum to meet the needs of our audience.

\end{document}